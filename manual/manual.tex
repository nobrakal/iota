\documentclass[10pt,a4paper]{article}
\usepackage[utf8]{inputenc}
\usepackage[T1]{fontenc}
\usepackage[english]{babel}
\usepackage{hyperref}

\newcommand\Iota{\textsc{Iota}}
\newcommand\Menhir{\textsc{Menhir}}
\newcommand\Odoc{\textsc{Odoc}}
\newcommand\Ocaml{\textsc{OCaml}}
\newcommand\Dune{\textsc{Dune}}

\usepackage{minted}
\newcommand{\ocaml}{\mintinline{ocaml}}

\usepackage{syntax}

\setcounter{tocdepth}{2}

\author{Alexandre Moine}
\title{Iota - Manual}

\begin{document}

\maketitle
\tableofcontents

\section{Introduction}
\Iota{} is a programming language targeting the logic developed in Husson's thesis \cite{husson}. More specifically, it targets a ``safe'' subset of Husson's logic (section 5). It emphasis a strong syntax and a type system, allowing to build only correct formulas.

\section{Semantics of the configuration file}
A \Iota{} program needs to be configured. For this purpose, users need to provide a configuration file separated in three parts:
\begin{enumerate}
\item The definition of constants.
\item The definition of types of objects.
\item The definition of predicates about these types.
\end{enumerate}

The full syntax is given in Section \ref{subsec:syntaxconfig}.
\subsection{The maxprof parameter}
All trees have \emph{bound} size, thus one must specify the maximum size of a tree. This is used in particular when compiling the \ocaml{TLink} predicate, see Section \ref{para:tlink}.

\subsection{Types}
Nodes, leaves and some operations on trees are \emph{typed}. These types are all defined in the configuration file, and are grouped in two families:
\begin{itemize}
\item Types without accessors. They will represent leaves.
\item Types with accessors (they will represent nodes). There is two type of accessors:
  \begin{itemize}
  \item A simple accessor composed by a name and a type.
  \item Multiple accessors, composed by a name, a number and a type. The syntax \ocaml{10 f to site} will allow the use, on a variable \ocaml{x}, of \ocaml{x.f:0}, \ocaml{x.f:1}, to \ocaml{x.f:9}. This is semantically equivalent to define 10 different accessors.
  \end{itemize}
  They both indicate the type of children and how to access it.
\end{itemize}

\subsection{Predicates}
\label{subsec:predicates}
Finally, one have to define predicates. A predicate gives information about a node of some type. See Section \ref{para:predicates} for more information.

\subsection{Example}
Here is a valid configuration file:
\begin{minted}[frame=lines]{ocaml}
let maxprof = 10

type site
type mol = m to mol | f to site | g to site
type bigmol = 5 tm to mol

static E about mol
dynamic Active about mol
dynamic Open about site
\end{minted}

This file declares three types:

\begin{itemize}
\item A leaf \ocaml{site}.
\item A node \ocaml{mol} with three possible children:
  \begin{itemize}
  \item An other node \ocaml{mol}.
  \item Two sites \ocaml{f} and \ocaml{g}.
  \end{itemize}
\item A node \ocaml{bigmol} with five possible children of type \ocaml{mol}.
\end{itemize}

\section{Semantics of a \Iota{} file}
Here is given semantics to the syntax fully described in Section \ref{subsec:syntaxprog}. A \Iota{} program is a file composed of four parts (the last three are exactly from Husson's thesis) :

\begin{enumerate}
\item A list of let-definitions.
\item A list of formulas (which may use the above definitions).
\item A list of formulas (which may use the above definitions) that must be ensured.
\item A list of formulas (which may use the above definitions) that must be maintained.
\end{enumerate}

\subsection{Propositional formulas}

\paragraph{Literals}
Formulas are about literals. A \Iota{} literal is either:
\begin{itemize}
\item A name.
\item A child of a literal.
\item The parent of a literal.
\end{itemize}

\paragraph{Predicates}
\label{para:predicates}
Predicates can be used to extract information about literals. Some predicates are bundled into the \Iota{} syntax itself:
\begin{itemize}
\item \ocaml{Eq} is a binary predicate expressing the fact that two literals are equals. It can be applied to literals of any types.
\item \ocaml{Link} is a binary predicate expressing the fact that two literals are linked. It can be applied to literals of any types.
\item \ocaml{TLink<t1,t2>} is a binary predicate expressing the fact that the two trees of the literals are linked. Its arguments must be of type \ocaml{t1} and \ocaml{t2}.
\item \ocaml{Has} is a unary predicate expressing the \emph{presence} of a literal. It can be applied to literal of any types.
\end{itemize}

The user can specify other predicates using the configuration file (see Section \ref{subsec:predicates}). Note that user-defined predicates are \emph{typed}, meaning that, once defined, they can be applied only to one type of literal.

\paragraph{Static vs Dynamic}
Predicates are separated into two categories: static and dynamic. Dynamic predicates can be prepended with the symbol \ocaml{+}. \ocaml{Eq}, \ocaml{Link} and \ocaml{Has} are dynamic.

\paragraph{Formulas}
Then you can build propositional formulas over these (fully applied) predicates using usual combiners \ocaml{not}, \ocaml{&&} and \ocaml{||}.

\paragraph{Examples}
Here are some examples using the previous configuration:
\begin{minted}[frame=lines]{ocaml}
Active(x);
Active(x) && not (Active (x.m));
Has(x) && (not (Open(x.t)) || +Link(x.t,y.g))
\end{minted}

\subsection{Toward first-order}
Then, one can construct first-order formulas using the two following structures:
\begin{itemize}
\item \ocaml{forall x guard(x,y) -> formula(x)}
\item \ocaml{exists x guard(x,y) && formula(x)}
\end{itemize}
Notice that each quantified variable is \emph{guarded}.

\paragraph{Guard}
A guard is a binary predicate involving exactly its two arguments. In \Iota{}, a guard is either \ocaml{Eq}, \ocaml{Link} or \ocaml{TLink}.

\subsection{Writing programs}
Now, we can wire up all these definitions using a standard let-syntax. Let-definitions are of the following form:
\begin{minted}{ocaml}
let e x = E(x)
let f h x y = (h x) && (e y)
\end{minted}
There is a few things to note:
\begin{itemize}
\item Definitions are not recursive and are parsed up-to-bottom. Thus, in the example, \ocaml{f} can use \ocaml{e} but the converse is false.
\item All free variables of right hand-side must be bound on the left hand-size, with some exceptions (see Section \ref{subsec:guardinfer}).
\item The language is \emph{high-order}, meaning that one definition can take another definition as an argument.
\end{itemize}

\subsection{General parts}
A program is optionally followed by two lists:
\begin{enumerate}
\item A list of formulas that must be \emph{ensured}.
\item A list of formulas that must be \emph{maintained}.
\end{enumerate}

These lists are composed of formulas are universally quantified and must respect Husson's criterion (see Section \ref{subsec:verification}).

\subsection{Examples}
\begin{minted}[frame=lines]{ocaml}
let activate_and_link x y = Link (x.f, y.g) && +Active(x)

in

Has(a) && (forall u Link(a,u) -> (E(u) || TLink<mol,mol>(x,u)));
(activate_and_link x y) || (activate_and_link x z)

ensure

Link (z.f, w.g) => (E(z) -> (E(w) && Open (z.f)))
\end{minted}


\section{The compilation back-end}
The compilation is divided into 5 steps:
\begin{enumerate}
\item Parsing.
\item Typechecking.
\item Compilation itself.
\item Guard inference.
\item Verification of the resulting structure.
\end{enumerate}

\subsection{Parsing}
The described syntax is parsed using the great \Menhir{}\cite{menhir}.

\subsection{Typechecking}
Programs are typechecked (using algorithm W) to ensure the validity of the file. Most of the inference is trivial. There is a challenge to infer the type of an expression using \ocaml{parent} since its usage introduce requirements about its argument (in the expression \ocaml{parent(x)}, \ocaml{x} must not be a root). This is bypassed using explicit typing of this operator.

\subsection{Compilation}
The compilation itself is divided into sub-steps.

\paragraph{Inlining}
All the definitions of \Iota{} are \emph{pure}, ie. without any effects, so one can always replace a call to a function by its body with proper substitution. This is the first thing the compiler do, since it allows to totally remove let-definitions. There are two sides effects:

\begin{itemize}
\item Unused definitions are not included in the result of the compilation. They still need to typecheck.
\item The size of the produced formula can be large since many things will be duplicated.
\end{itemize}

\paragraph{Compilation of TLink}
\label{para:tlink}
The \ocaml{Tlink} predicate is not included in the target language, so we need to compile it to a big disjunction of cases. Note that, thanks to the type system, the compilation of \ocaml{TLink} produces only valid (with respect to the type system) possibilities.

\paragraph{Optimization}
The compiler can now do some optimization. It actually does only one, by replacing expressions of the form:
\begin{minted}{ocaml}
parent<t1,t2>(x.f)
\end{minted}
by
\begin{minted}{ocaml}
x
\end{minted}
Note that this optimization is legal in the sense that, since the program typechecks, we know that \ocaml{x.f} is a valid object.

\subsection{Guard inference}
\label{subsec:guardinfer}
When writing a definition, it is convenient to see free variables as \emph{existentially} quantified ones. The problem is that, in \Iota{}, every quantified variable must be guarded. Thus we need to infer a guard for free variables in let-definitions. Some heuristics are used and one can write a definition such as:
\begin{minted}{ocaml}
let f x = Eq(x,z) && Link(x,z)
\end{minted}
which will be compiled to:
\begin{minted}{ocaml}
let f x = exists z Eq(x,z) && Link(x,z)
\end{minted}

\paragraph{Normal form of formulas}
Any formula can be put in a \emph{normal form}, where negation is only present on leaves. This is done here. Note that it was not possible to do it before, since we need to first infer guards to be able to negate them properly.

\subsection{Verification}
\label{subsec:verification}
The resulting structure must satisfy some properties. We must ensure that every guard is indeed a well-formed guard, and that the two parts ``ensure'' and ``maintain'' are well-formed. This pass only checks if these properties are respected.

\section{About the compiler}
\subsection{Compiling}
The compiler is written using \Ocaml{}\cite{ocaml}. It was designed to be built with \Dune{}\cite{dune}. It is itself separated into two components:

\begin{enumerate}
\item A library, located in \verb|lib/| were all the work is made. Its entry-point is \verb|lib/main.ml|. It is fully documented using \Odoc{}\cite{odoc}. Online documentation is available at: \url{https://nobrakal.github.io/iota/iota/}, or can be built using:
\begin{minted}{shell}
dune build @doc
\end{minted}
\item A front-end, located in \verb|bin/|, providing an executable able to launch the compilation of a file. It can be built using:
\begin{minted}{shell}
dune build bin/main.exe
\end{minted}
  The produced executable takes as argument the configuration file and the file itself. It outputs the result of the compilation or prints an error.
\end{enumerate}

\subsection{Testing}
Some efforts were made to be able to test the compiler. The test-suite is located in \verb|test/|. It can be ran using:
\begin{minted}{shell}
dune runtest
\end{minted}
For now, it ensures that files that must compile are indeed compiling, and files that must fail to compile are indeed failing.

\section{Syntax of the language}
Nota Bene:
\begin{itemize}
\item \verb|[<rule>]| symbolise a list of \verb|<rule>|.
\item \verb|[<rule> // <sep>]| symbolise a list of \verb|<rule>| separated by \verb|<sep>|.
\end{itemize}

Comments à-la-OCaml can be used everywhere.

\subsection{Syntax of configuration file}
\label{subsec:syntaxconfig} Here is a BNF syntax for the configuration file.
\begin{grammar}
  <constants> ::=
  \lit{let} \lit{maxprof} \lit{=} <int>

  <accessor> ::=
  <lident> \lit{to} <type>
  \alt <int> <lident> \lit{to} <type>

  <typedef> ::=
  \lit{type} <lident>
  \alt \lit{type} <lident> \lit{=} [<accessor> // \lit{|}]

  <preddef> ::=
  \lit{static} <hident> \lit{about} <lident>
  \alt \lit{dynamic} <hident> \lit{about} <lident>

  <config> ::=
  <constants> [<typedef>] [<preddef>]
\end{grammar}

\subsection{Syntax of programs}
\label{subsec:syntaxprog} Here is a BNF syntax for a \Iota{} program.
\begin{grammar}
  <lident> ::= [a-z][a-zA-Z0-9]*

  <hident> ::= [A-Z][a-zA-Z0-9]*

  <type> ::= <lident>

  <term> ::= <lident>
  \alt `parent<' <type> `,' <type> `>(' <term> `)'
  \alt <term> `.' <lident> `:' <int>
  \alt <term> `.' <lident>

  <guard> ::= `Eq(' <term> `,' <term> `)'
  \alt `Link(' <term> `,' <term> `)'
  \alt `TLink<' <type> `,' <type> `>(' <term> `,' <term> `)'

  <predicate> ::= <guard>
  \alt `Has(' <term> `)'
  \alt <hident> `(' <term> `)'

  <formula> ::= <predicate>
  \alt `+'<predicate>
  \alt `not' <formula>
  \alt <formula> `&&' <formula>
  \alt <formula> `||' <formula>
  \alt `(' <formula> `)'

  <safe> ::= <formula>
  \alt <lident>
  \alt <safe> `&&' <safe>
  \alt <safe> `||' <safe>
  \alt `forall' <lident> <guard> `->' <safe>
  \alt `exists' <lident> <guard> `&&' <safe>
  \alt <safe> <safe>
  \alt `(' <safe> `)'

  <letdef> ::= \lit{let} <lident> [<lident>] \lit{=} <safe>

  <general> ::=
  <guard> \lit{->} <general>
  \alt \lit{=>} <formula>

  <program> ::=
  [<letdef>] \lit{in} [<safe> // \lit{;}] \lit{ensure} [<general> // \lit{;}] \lit{maintain} [<general> // \lit{;}]

\end{grammar}

\bibliographystyle{plain}
\bibliography{publications}

\end{document}
%%% Local Variables:
%%% mode: latex
%%% TeX-master: t
%%% End:
