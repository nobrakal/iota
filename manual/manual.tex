\documentclass[10pt,a4paper]{article}
\usepackage[utf8]{inputenc}
\usepackage[T1]{fontenc}
\usepackage[english]{babel}
\usepackage{hyperref}

\usepackage{minted}
\newcommand{\ocaml}{\mintinline{ocaml}}

\usepackage{syntax}

\setcounter{tocdepth}{2}

\author{Alexandre Moine}
\title{Iota - Manual}

\begin{document}

\maketitle
\tableofcontents

\section{Introduction}
Iota is a programming language.

\section{Syntax of the language}
Nota Bene:
\begin{itemize}
\item \verb|[<rule>]| symbolise a list of \verb|<rule>|.
\item \verb|[<rule> // <sep>]| symbolise a list of \verb|<rule>| separated by \verb|<sep>|.
\end{itemize}

Comments à-la-OCaml can be used everywhere.

\subsection{Syntax of programs}

\begin{grammar}
  <lident> ::= [a-z][a-zA-Z0-9]*

  <hident> ::= [A-Z][a-zA-Z0-9]*

  <type> ::= <lident>

  <term> ::= <lident>
  \alt `parent<' <type> `,' <type> `>(' <term> `)'
  \alt <term> `.' <lident>

  <guard> ::= `Eq(' <term> `,' <term> `)'
  \alt `Link(' <term> `,' <term> `)'
  \alt `TLink<' <type> `,' <type> `>(' <term> `,' <term> `)'

  <predicate> ::= <guard>
  \alt `Has(' <term> `)'
  \alt <hident> `(' <term> `)'

  <formula> ::= <predicate>
  \alt `+'<predicate>
  \alt `not' <formula>
  \alt <formula> `&&' <formula>
  \alt <formula> `||' <formula>
  \alt `(' <formula> `)'

  <safe> ::= <formula>
  \alt <lident>
  \alt <safe> `&&' <safe>
  \alt <safe> `||' <safe>
  \alt `forall' <lident> <guard> `->' <safe>
  \alt `exists' <lident> <guard> `&&' <safe>
  \alt <safe> <safe>
  \alt `(' <safe> `)'

  <letdef> ::= \lit{let} <lident> [<lident>] \lit{=} <safe>

  <general> ::=
  <guard> \lit{->} <general>
  \alt \lit{=>} <formula>

  <program> ::=
  [<letdef>] \lit{in} [<safe> // \lit{;}] \lit{ensure} [<general> // \lit{;}] \lit{maintain} [<general> // \lit{;}]

\end{grammar}

\subsection{Syntax of configuration file}
\begin{grammar}
  <constants> ::=
  \lit{let} \lit{maxprof} \lit{=} <int>

  <typedef> ::=
  \lit{type} <lident>
  \alt \lit{type} <lident> \lit{=} [<lident> \lit{to} <type> // \lit{|} ]

  <preddef> ::=
  \lit{static} <hident> \lit{about} <lident>
  \alt \lit{dynamic} <hident> \lit{about} <lident>

  <config> ::=
  <constants> [<typedef>] [<preddef>]
\end{grammar}

\subsection{Semantics of the configuration file}
\subsection{The maxporf parameter}
All trees have \emph{bound} size, thus one must specify the maximum size of a tree.

\subsection{Types}
Nodes, leaves and some operations on trees are \emph{typed}.
\begin{itemize}
\item Types without accessors. They will represent leaves.
\item Types with accesors. They will represent nodes. Each accessor is composed by a name and a type.
  This indicates the type of the child and how to access it.
\end{itemize}

\subsection{Predicates}
Finally, one have to define predicates. A predicate gives information about a node of some type.

\subsection{Example}
Here is a valid configuration file:
\begin{minted}[frame=lines]{ocaml}
let maxprof = 10

type site
type mol = m to mol | f to site | g to site

static E about mol
dynamic Active about mol
dynamic Open about site
\end{minted}

This file declares two types:

\begin{itemize}
\item A leaf \ocaml{site}.
\item A node \ocaml{mol} with three possible children:
  \begin{itemize}
  \item An other node \ocaml{mol}.
  \item Two sites \ocaml{f} and \ocaml{g}.
  \end{itemize}
\end{itemize}

\section{Semantics of a iota file}

\subsection{Propositional formulas}
All iota files are \emph{in fine} a list of simple formulas.

\paragraph{Literals}
Formulas are about literals. A iota literal is either:
\begin{itemize}
\item A name.
\item A child of a literal.
\item The parent of a literal.
\end{itemize}

\paragraph{Predicates}
Predicates can be used to extract information about literals. Some predicates are bundled into the iota syntax itself:
\begin{itemize}
\item \ocaml{Eq} is a binary predicate expressing the fact that two literals are equals. It can be applied to literals of any types.
\item \ocaml{Link} is a binary predicate expressing the fact that two literals are linked. It can be applied to literals of any types.
\item \ocaml{TLink<t1,t2>} is a binary predicate expressing the fact that the two trees from the literals are linked. Its arguments must be of type \ocaml{t1} and \ocaml{t2}.
\item \ocaml{Has} is a unary predicate expressing the \emph{presence} of a literal. It can be applied to literal of any types.
\end{itemize}

The user can specify other predicates using the configuration file. Note that user-defined predicates are \emph{typed}, meaning that, once defined, they can be applied only to one type of literal.

\paragraph{Static vs Dynamic}
Predicates are separated into two categories: static and dynamic. Dynamic predicates can be prepended with the symbol \ocaml{+}. \ocaml{Eq}, \ocaml{Link} and \ocaml{Has} are dynamic.

\paragraph{Formulas}
Then you can build propositional formulas over these (fully applied) predicates using usual combinators \ocaml{not}, \ocaml{&&} and \ocaml{||}.

\paragraph{Examples}
Here is some examples using the previous configuration:
\begin{minted}[frame=lines]{ocaml}
Active(x);
Active(x) && not (Active (x.m));
Has(x) && (not (Open(x.t)) || +Link(x.t,y.g))
\end{minted}

\subsection{Toward first-order}
Then, one can construct first-order formulas using the two following structures:
\begin{itemize}
\item \ocaml{forall x, guard(x,y) -> formula(x)}
\item \ocaml{exists x, guard(x,y) && formula(x)}
\end{itemize}
Notice that each quantifies variable is \emph{guarded}.

\subsection{Writing programs}
Now, we can wire up all these definitions using a standard let-syntax.

\section{The compilation back-end}
The compilation is divided into 4 steps:
\begin{enumerate}
\item Parsing.
\item Typechecking.
\item Compilation itself.
\item Verification of the resulting structure.
\end{enumerate}

\subsection{Parsing}
The described syntax is parsed using the great Menhir: \url{http://gallium.inria.fr/~fpottier/menhir/}.

\subsection{Typechecking}
Programs are typechecked (using algorithm W) to ensure the validity of the file.

\subsubsection{Guard inference}

\subsection{Compilation}
The compilation itself is divided into sub-steps.

\paragraph{Inlining}

\paragraph{Compilation of TLink}

\paragraph{Normal form of formulas}

\paragraph{Optimization}

\subsection{Verification}
The resulting structure must satisfies some properties.

\end{document}
%%% Local Variables:
%%% mode: latex
%%% TeX-master: t
%%% End:
